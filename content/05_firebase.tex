%! Author = Philipp Emmenegger
%! Date = 12/07/2021

\section{Firebase}
Läuft in der Google Cloud Platform.
Hauptfokus von Firebase sind Mobile- und Web-Apps.

\subsection{Firebase Authentication}
Backend Services für Authentifizierung und einfache Userverwaltung.
SDKs für diverse Plattformen.
Vorgefertigte UI Libraries

\subsection{Firebase Hosting}
Einfaches Hosting für \textbf{statischen Content}:
Immer per HTTPS ausgeliefert.
Automatisches Caching in CDNs.
\textbf{Dynamischer Content}: nur über \textbf{Cloud Function}, wenn das nicht reicht:
PaaS: Google App Engine,
Docker: Google Container oder Kubernetes Engine.

\subsubsection{Serverless Computing}
\textbf{Cloud Provider verwaltet Functions:}
Deployment geschieht on-demand.
Plattform bestimmt die Parallelisierung.
Entwickler hat keine Kontrolle über laufende Instanzen.
Funktionen sind Stateless.
Abgerechnet werden Aufrufe und Laufzeit der Funktion.
\textbf{Limitationen:} Ausführungszeit / Memory begrenzt.
Teilweise hohe Latenz.

\subsubsection{Firebase Cloud Functions}
\textbf{Anwendungszenarien:} Code als Reaktion auf einen Event ausführen,
Administration (Cron Jobs), REST API für Mobile und SPAs zur Verfügung stellen.

\subsubsection{Cloud Firestore}
NoSQL, document-oriented database.
DB besteht aus mehreren Collections mit Documents.
Document ist ein JSON-Objekt.
Document kann Collections beinhalten.
Vergleichbar mit MongoDB.
Stark eingeschränkte Queries (keine Volltextsuche).

\subsubsection{NoSQL Many-To-Many}
\begin{itemize}
    \item Wie in relationaler Datenbank mit Assoziationstabelle
    \begin{itemize}
        \item Kein kopieren von Daten
        \item Komplexere Abfragen, keine Joins im Firestore
    \end{itemize}
    \item Oder Daten kopieren und einbetten
\end{itemize}
\textbf{Kopieren der Daten:} muss kein Nachteil sein.
Preisänderung eines Produktes hat keinen Einfluss auf vergangene Bestellungen.